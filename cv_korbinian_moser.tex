\documentclass[a4paper, ]{article}

\usepackage[margin=2.5cm]{geometry}
\usepackage{titling}
\usepackage{hyperref}
\usepackage{titlesec}
\usepackage{fontawesome}

\titleformat{\section}
{\Large\bfseries}
{}
{0em}
{}[\titlerule]

\titlespacing{\subsection}{0em}{.25em}{.5em}

\hypersetup{
    colorlinks=true,      
    urlcolor=blue
}

\renewcommand{\maketitle}{
    \begin{center}
        \Large\thetitle\vspace{0.25cm}\\
        {\huge\bfseries\theauthor}
    \end{center}
}

\newenvironment{cventry}[2]
{   
    \setlength{\tabcolsep}{1.25em}
    \hypersetup{urlcolor=black}
    \begin{center}\hspace{-1.25cm}\begin{tabular}{p{0.25\linewidth}|p{0.65\linewidth}}
    \raggedleft\scshape #1 & \bfseries#2 \vspace{0.1cm}\\ & 
}
{\end{tabular}\end{center}}

\newenvironment{cventrynoheading}[1]
{
    \setlength{\tabcolsep}{1.25em}
    \begin{center}\hspace{-1.25cm}\begin{tabular}{p{0.25\linewidth}|p{0.65\linewidth}}
    \raggedleft\scshape #1 & 
}
{\end{tabular}\end{center}}

\newcommand{\newentryline}{\\&}


\begin{document}
    
\title{R\'esum\'e}
\author{Korbinian Moser}

\maketitle

\section{Personal Info}

\begin{center}
\setlength{\tabcolsep}{0.1em}
\small
\begin{tabular}{p{1em}p{0.4\linewidth}cp{0.5\linewidth}}
    \faEnvelopeO&: \href{mailto:me@example.com}{\nolinkurl{korbi@korbinian-moser.de}} & 
    \faGithub&: \href{https://github.com/korbi98}{\nolinkurl{github.com/korbi98}} \vspace{0.25cm}\\  
    \faGlobe&: \url{https://korbinian-moser.de} & 
    \(\mathrm{R^G}\)&: \href{https://www.researchgate.net/profile/Korbinian_Moser}{\nolinkurl{researchgate.net/profile/Korbinian_Moser}}
\end{tabular}
\end{center}
\noindent
Currently pursuing a Masters degree in physics at Ludwig-Maximilans-University
Munich. My main interests lie within applied research, especially nuclear fusion, 
plasma physics, computational physics and solid state physics.

\section{Experience}
\begin{cventry}{09/2021 -- 01/2022}
    {Swiss Plasma Center EPFL}
    Research project regarding data analysis of AXUV measurements of the RADCAM 
    diagnostic at TCV
\end{cventry}

\begin{cventry}{11/2020 -- 02/2021}
    {Max Planck Institute for Plasma Physics}
    Scientific assistant, working on emission tomography for bolometer 
    measurements via Gaussian process regression
\end{cventry}

\begin{cventry}{04/2018 -- 01/2019}
    {InTech GmbH, Garching by Munich}
    Working student, working on train simulation in Matlab/Simulink within the 
    \href{https://www.researchgate.net/project/TORPA-Toolbox-for-Optimal-Railway-Propulsion-Architectures}{TORPA} 
    project for estimating efficiency of different propulsion architectures 
\end{cventry}

\begin{cventry}{2013 -- 2014}
    {Technical Secondary School Wasserburg \& EAS GmbH}
    20 weeks of internships in metalworking and electrical installation
    during 11th grade
\end{cventry}


\section{Education}

\begin{cventry}{2020 -- present}
    {M.Sc. in Physics}
    \scshape Ludwig-Maximilans-University Munich
\end{cventry}

\begin{cventry}{09/2021 -- 02/2022}
    {Erasmus Exchange}
    \scshape EPFL Lausanne, Switzerland
\end{cventry}

\begin{cventry}{2017 -- 2020}
    {B.Sc. in Physics}
    \scshape Ludwig-Maximilans-University Munich \newentryline Final grade 1.4, 
    Bachelor thesis: \slshape Gaussian Processes for Emission Tomography at 
    ASDEX Upgrade \href{https://gitlab.mpcdf.mpg.de/komo/bachelorarbeit}{\normalfont\faLink}
\end{cventry}

\begin{cventry}{2015 -- 2016}
    {Abitur}
    \scshape  Technical Secondary School Munich East
\end{cventry}

\begin{cventry}{2013 -- 2015}
    {Advanced Technical College Certificate}
    \scshape Technical Secondary School Wasserburg
\end{cventry}


\section{Skills}

\begin{cventrynoheading}{Languages}
    German (native), English (fluent), French (beginner)
\end{cventrynoheading}

\begin{cventrynoheading}{Programming}
    Python including the SciPy stack, Java, Android app development 
    in Java and Kotlin, basic C/C++ knowledge, Matlab and Simulink
\end{cventrynoheading}

\begin{cventrynoheading}{Other technical skills}
    Git version control, \LaTeX\ and Markdown, advanced Linux/Unix knowledge, 
    basic electronic projects with Arduino/RaspberryPi, FDM 3D printing
\end{cventrynoheading}


\section{Projects}

\begin{cventry}{2019}
    {Simple Budget \href{https://github.com/korbi98/Simple-Budget}{\normalfont\faGithub}}
    Android app written in Kotlin for managing expenses. It has an intuitive 
    user interface, custom expense categories and can track budget limits 
    for individual categories.
\end{cventry}

\begin{cventry}{2018}
    {TORPA \href{https://www.researchgate.net/project/TORPA-Toolbox-for-Optimal-Railway-Propulsion-Architectures}{\normalfont\faLink}}
    {\scshape Toolbox for Optimal Railway Propulsion Architectures}
    Contributed to the project by developing a train simulation 
    in Matlab/Simulink that can estimate the energy consumption of 
    different propulsion architectures on given train routes.
\end{cventry}

\begin{cventry}{2018}
    {Simple Sudoku \href{https://github.com/korbi98/Simple-Sudoku}{\normalfont\faGithub}}
    Sudoku game for Android written in Java
\end{cventry}

\begin{cventry}{2018}
    {Simple ToDoList \href{https://github.com/korbi98/Simple-ToDoList}{\normalfont\faGithub}}
    Simple ToDo-list app for Android written in Java
\end{cventry}


\section{Volunteer Work}

\begin{cventry}{2011 -- present}{Voluntary Fire Brigade}
    Active member of local voluntary fire brigade, \newentryline 
    Since 2018 treasurer of the local fire brigade
\end{cventry}

\begin{cventrynoheading}{2020}
    Candidacy for municipal council of Moosach
\end{cventrynoheading}


\section{Other}

\begin{cventry}{2018 -- present} {Deutschlandstipendium scholarship}
    Since April 2018 I am part of the Deutschlandstipendium, a scholarship 
    funded partially by the German government and private sponsors for 
    committed and high-achieving students in Germany.
\end{cventry}

\begin{cventry}{09/2016 -- 04/2017}{Work and Travel in New Zealand}
    Besides traveling I did different farm work and improved my English skills.
\end{cventry}


\end{document}

